\newpage
{
	\justifying
\section{Conceptualizacion}
\subsection{Analisis preliminar de los sistemas de operación de la organización}
\subsubsection{¿Cómo opera actualmente el estacionamiento?}
El sistema actual requiere que el usuario solicite información sobre cómo acceder 
a los beneficios del estacionamiento en la unidad. Para registrarse, el
usuario debe proporcionar al administrador los siguientes documentos, los cuales
cumplen roles específicos dentro del proceso de autorización:
\begin{enumerate}
\item{\textbf{Datos del alumno/Administrativo:}  Permite identificar si el solicitante
	es miembro de la comunidad universitaria y verificar su vínculo con la
	institución, condición esencial para acceder al estacionamiento.}
\item{\textbf{Tarjeta de circulación(actualizada):} Verifica que el vehículo tiene
	permiso legal para circular, y que la documentación del mismo está en
	regla. Esto garantiza que sólo se autoricen vehículos con papeles válidos.}
\item {\textbf{Tira de materias:}Se utiliza para confirmar el horario de clases del solicitante, lo cual
	 ayuda a asignar el estacionamiento a quienes efectivamente
	requieren el servicio en horarios específicos. }
\item{\textbf{Telefonos de contacto:}Facilita la comunicación rápida con el usuario
	en caso de cualquier emergencia o situación que requiera su atención in-
	mediata dentro del estacionamiento.}
\item{\textbf{Datos del vehículo:} Estos datos son esenciales para identificar el vehículo 
	autorizado, permitiendo un control adecuado y evitando que vehículos 
	no registrados accedan al estacionamiento.  }
\end{enumerate}
Con esta información, el administrador puede validar el cumplimiento de los
requisitos y registrar a los usuarios autorizados. Este proceso actualmente es
manual, lo que genera in eficiencias, como el riesgo de pérdida o deterioro de
documentos y la falta de un sistema organizado de almacenamiento de datos.
\subsubsection{¿Qué procesos se siguen actualmente, aunque sean manuales?}
Los procesos manuales que se llevan a cabo incluyen:
\begin{enumerate}

\item Verificación de los documentos entregados por el usuario.
\item  Registro manual de los datos en un archivo físico o en una base de datos
no estructurada.
\item Archivo de los documentos, los cuales son almacenados en carpetas físicas.
\end{enumerate}
\subsubsection{¿Cuáles son los casos más comunes que se presentan?}
Los casos que suelen ocurrir en la operación del estacionamiento incluyen:
\begin{enumerate}
\item Solicitudes de renovación de permisos, especialmente cuando la tarjeta de
circulación debe actualizarse.
\item  Pérdida de documentos, lo cual puede requerir que el usuario presente
nuevamente la información.
\item Incidencias de vehiculos no autorizados
\end{enumerate}
\subsection{Diagnostico de la situación de los sistemas de operación de la organización}
La gestión de solicitudes y los procesos actuales presentan varios problemas que
afectan la eficiencia operativa del estacionamiento
\begin{enumerate}
\item{ \textbf{Falta de control centralizado de datos:} Al no contar con una base
	de datos estructurada, la recuperación de información es lenta y poco
	confiable. Esto puede llevar a errores en la autorización de acceso.}
\item{ Riesgo de pérdida o deterioro de documentos: Almacenar documentos 
	físicamente aumenta el riesgo de que se extravíen o se deterioren,
	lo que impacta la confiabilidad del sistema.
}
\item {\textbf{Procesos lentos y dependientes de personal: }La verificación y el
	registro manual de cada usuario son tareas que requieren mucho tiempo y
	son susceptibles a errores humanos, lo cual reduce la eficiencia.}

\item {\textbf{Dificultad en la actualización de información: }La actualización de
documentos importantes, como la tarjeta de circulación, no es ágil, ya que
depende de un proceso de validación manual.}
\item{\textbf{Falta de organización en el almacenamiento de documentos:} La
	dependencia de archivos físicos y el escaso control en su orden y disponibilidad
	 generan problemas de espacio y dificultad para acceder a información
	histórica.}
\end{enumerate}
Este diagnóstico sugiere que la implementación de un sistema digital y centralizado 
podría mejorar significativamente la eficiencia y confiabilidad del pro-
ceso de administración del estacionamiento.
\subsection{Propuesta de sistema de información computacional que solucione la problemática detectada en la organización}

La propuesta consiste en la creación e implementación de un sistema web que centralice y optimice el proceso de administración del estacionamiento. Este sistema permitirá que tanto el administrador del estacionamiento como los usuarios (alumnos y personal administrativo) gestionen los trámites de manera más eficiente y organizada, reduciendo el papeleo y los errores, así como acelerando los tiempos de respuesta.

\subsubsection{Funciones para el administrador del estacionamiento}

El sistema proporcionará al administrador herramientas clave para la gestión de solicitudes y el control de los usuarios autorizados. A continuación, se listan las funcionalidades, ordenadas de mayor a menor prioridad:

\begin{enumerate}
	\item \textbf{Gestión de solicitudes:} El administrador podrá recibir, revisar y responder solicitudes de acceso directamente desde el sistema, eliminando la necesidad de documentos físicos y agilizando el flujo de trabajo.
	\item \textbf{Almacenamiento de datos en una base de datos centralizada:} Toda la información de los usuarios (personal y del vehículo) se almacenará de forma estructurada y segura, permitiendo una consulta rápida y confiable.
	\item \textbf{Organización y búsqueda eficiente de registros:} El sistema contará con opciones de filtrado y búsqueda para acceder fácilmente a información específica y mejorar la organización general.
	\item \textbf{Notificaciones automáticas por correo electrónico:} El sistema notificará al administrador sobre nuevas solicitudes pendientes, lo que permitirá una atención oportuna.
	\item \textbf{Reducción de papeleo y espacio físico:} Al digitalizar los trámites, se eliminará la necesidad de archivar documentos físicos, optimizando el espacio y reduciendo riesgos de extravío o deterioro.
\end{enumerate}

\subsubsection{Beneficios para los alumnos y personal administrativo}

El sistema también busca mejorar la experiencia del usuario mediante un proceso digital cómodo y accesible. Las funcionalidades están ordenadas por prioridad:

\begin{enumerate}
	\item \textbf{Solicitud en línea desde cualquier dispositivo:} Los usuarios podrán realizar la solicitud desde cualquier lugar y dispositivo con acceso a internet, sin necesidad de acudir presencialmente.
	\item \textbf{Respuesta rápida a solicitudes:} El sistema enviará notificaciones automáticas informando a los usuarios cuando su solicitud haya sido atendida.
	\item \textbf{Reducción de tiempos de espera:} Al digitalizar el trámite, se eliminarán las filas y las demoras en la entrega de documentos físicos.
	\item \textbf{Generación de comprobante digital (váucher):} Los usuarios podrán descargar e imprimir su comprobante de acceso desde casa, simplificando el ingreso.
	\item \textbf{Opciones de pago en línea:} En caso de tarifas por mantenimiento o mejoras, los usuarios podrán realizar pagos directamente desde el sistema.
\end{enumerate}

Con este sistema se espera optimizar significativamente la operación del estacionamiento, brindando una experiencia más rápida, cómoda y segura tanto para el administrador como para los usuarios. Además, la centralización digital de los datos contribuirá a una gestión más ágil y segura.




}