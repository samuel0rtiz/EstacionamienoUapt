{
\newpage
\section{Introducción}
En la actualidad, la gestión eficiente de recursos es un aspecto fundamental
para el buen funcionamiento de cualquier organización. En el contexto educativo, donde 
las instalaciones deben atender a un número creciente de estudiantes
y personal, el manejo adecuado de los espacios de estacionamiento se convierte
en un reto significativo. La falta de un sistema adecuado para la gestión de
solicitudes y asignación de espacios puede generar inconvenientes tanto para los
usuarios como para los administradores, resultando en largas filas, pérdida de
documentos y un control ineficiente de la información.\\

Este documento presenta el desarrollo de un sistema de información computacional 
que busca optimizar la gestión del estacionamiento en la Unidad
Académica Profesional Tianguistenco de la Universidad Autónoma del Estado
de México. El objetivo principal es facilitar el acceso y la organización de la in-
formación relacionada con las solicitudes de estacionamiento, garantizando una
experiencia más ágil y eficiente para los beneficiarios.\\

El sistema propuesto se fundamenta en el análisis preliminar de las operaciones 
actuales y los problemas identificados en la gestión de solicitudes. A
través de un enfoque en la centralización de datos y la reducción del uso de papel, se espera 
mejorar la comunicación entre los administradores y los usuarios,
así como optimizar el tiempo de respuesta a las solicitudes.\\

A lo largo de este documento, se detallarán las características del sistema
propuesto, su análisis de factibilidad, así como las especificaciones técnicas necesarias 
para su implementación exitosa.\\
\subsection{Proposito}
El propósito de este documento es proporcionar una descripción detallada del
sistema de información computacional propuesto para la gestión del estacionamiento
en la Unidad Académica Profesional Tianguistenco de la Universidad Autónoma
del Estado de México. Este sistema tiene como objetivo principal mejorar
la eficiencia en la gestión de solicitudes y la asignación de espacios de estacionamiento, facilitando 
la interacción entre los administradores y los beneficiarios.
Específicamente, este documento busca:\\

\begin{enumerate}


\item { \textbf{Definir los Requerimientos del Sistema}: Establecer las necesidades funcionales 
	y no funcionales que el sistema debe cumplir para garantizar su
	efectividad y satisfacción del usuario.}
	
\item {\textbf{Identificar Problemas Actuales}: Analizar las deficiencias del sistema
	actual, resaltando las in eficiencias y los riesgos asociados con la gestión
	manual de solicitudes y documentos.}
\item {\textbf{Presentar la Propuesta de Solución: }Describir las características y
	funcionalidades del sistema propuesto, enfatizando su capacidad para centralizar 
	la información, automatizar procesos y mejorar la comunicación.}

\item {\textbf{Establecer un Plan de Implementación: }Proporcionar un enfoque
estratégico para la implementación del sistema, incluyendo fases de desarrollo, 
recursos necesarios y cronograma.}

\item{\textbf{Facilitar la Evaluación de Factibilidad: }Evaluar la viabilidad técnica y
	 económica del sistema, asegurando que la solución propuesta sea
sostenible y beneficiosa para la organización.}
\end{enumerate}
A través de este documento, se pretende establecer una base sólida para la toma
de decisiones en la implementación del sistema, asegurando que se alineen con
los objetivos estratégicos de la unidad académica y se brinde un servicio de
calidad a todos los usuarios.
\subsection{Ambito de sistema }
\begin{itemize}
\item{\textbf{Nombre del sistema  :}EstacionaUAPT}
\item {\textbf{Lo que el sistema hará :} Recibir datos de los usuarios , tener una base
	de datos de los usuarios,aceptar solicitudes,rechazar solicitudes,exportar
	a excel los datos de los usuarios ,realizar filtros , posibilidad de editar el
	talón de estacionamiento.}
\item {\textbf{Lo que el sistema no hará :}Tomar los datos de los alumnos no han
	proporcionado y generar los boletos de estacionamiento,realizar registros
	automáticos , aceptar y rechazar automáticamente solicitudes,}
	\item{\textbf{Beneficios :}
	\begin{enumerate}
\item Eficiencia de la gestión de solicitudes
\item Ahorro de tiempo para los usuarios
\item Respuesta rápida
\item Centralización y seguridad de la información
\item Reducción del uso de papel
\item Mejora en la toma de decisiones
\item Esca labilidad y adaptabilidad
		\end{enumerate}}
\item {\textbf{Objetivos} 
\begin{enumerate}
\item  Automatizar el proceso de solicitud y asignación de espacios de estacionamiento.
\item Centralizar la información en una base de datos segura y accesible.
\item Reducir los tiempos de espera y mejorar la experiencia del usuario.
\item Eliminar el uso de documentos físicos para contribuir a la sostenibilidad ambiental.
\item Facilitar el acceso a la información y el monitoreo de solicitudes en
tiempo real.
\item Permitir la integración con otros sistemas y servicios institucionales
en el futuro.
\item Mejorar la organización y el control de datos para apoyar la toma de
decisiones
	\end{enumerate}}	
\item{\textbf{Metas}
	\begin{enumerate}
\item Implementar un sistema web de gestión de estacionamiento en un
		periodo de seis meses.
\item Lograr que al menos el 80\% de las solicitudes sean procesadas digitalmente en el primer año.
\item Reducir en un 50\% el tiempo promedio de respuesta a las solicitudes
		de estacionamiento.
\item Disminuir el uso de papel en la gestión de estacionamiento en al
		menos un 90\%.
\item Asegurar que el 100\% de los datos de usuarios y vehículos estén almacenados 
de manera segura y centralizada.
\item Integrar el sistema con el correo institucional para notificaciones automáticas dentro del primer trimestre de implementación.
\item Realizar un análisis de satisfacción de usuarios después del primer
		año de operación para identificar áreas de mejora.
		

	\end{enumerate}
	
}
\end{itemize}
\subsection{ Definiciones , Acrónimos y Abreviaturas}
\subsubsection{Definiciones}
\begin{itemize}
	 \item{\textbf{Usuario: }Persona que solicita acceso al sistema de estacionamiento, ya
	sea estudiante, docente o personal administrativo.}
\item {\textbf{Administrador: }Persona responsable de gestionar y autorizar las solicitudes de acceso al estacionamiento.}
\item {\textbf{EstacionaUAPT: }Sistema de gestión de estacionamiento de la
	Unidad Académica Profesional Tianguistenco.}
\item{ \textbf{Solicitud de acceso: }Proceso mediante el cual un usuario registra sus
	datos y documentos para obtener permiso para estacionarse en las instalaciones.}

\item {\textbf{Tarjeta de circulación:} Documento legal que certifica que un vehículo
	está autorizado para circular}
	\item { \textbf{Base de datos centralizada: }Almacenamiento estructurado y seguro
	de la información del sistema, accesible solo por usuarios autorizados.}
	\item {\textbf{Backend:}Parte de un sistema informático o aplicación web que se encarga de la lógica interna, el procesamiento de datos y la comunicación con bases de datos y otros servicios. }

\end{itemize}


\subsubsection{Acrónimos}
\begin{itemize}
\item { \textbf{UAPT:} Unidad Académica Profesional Tianguistenco.}
\item {\textbf{UI:} Interfaz de Usuario (\emph{User Interface}).}
\item {\textbf{API:} Interfaz de Programación de Aplicaciones (\emph{Application Programming Interface}).}
\end{itemize}
\subsubsection{Abreviaturas}
\begin{itemize}
\item{ \textbf{Req.:} Requisito.}
\item{\textbf{Est.:} Estacionamiento.}
\item{ \textbf{Sist.:} Sistema.}
\item {\textbf{Doc.: } Documento.}
\subsection{Referencias}
\begin{enumerate}
	\item IEEE Std 830-1998, *IEEE Recommended Practice for Software Requirements Specifications*.
	\item Reglamento de uso del estacionamiento de la Universidad [Universidad Autonoma Del Estado de Mexico].
	\item Entrevistas realizadas al personal de seguridad y administración del estacionamiento (fecha: marzo 2025).
	\item Documento “Requerimientos funcionales preliminares del sistema de estacionamiento”, versión 1.0.
\end{enumerate}


\end{itemize}
\subsection{ Vision general del documento}
El presente documento de especificación de requisitos de software(ERS)se
encuentra organizado en varias secciones que describen de manera estructurada
los aspectos necesarios para el desarrollo e implementación del sistema de gestión
del estacionamiento UAPT. A continuación, se ofrece una breve descripción de
los contenidos de cada sección:
\begin{itemize}
	  \item {\textbf{Sección 1: Conceptualización.} Presenta un análisis del sistema actual de operación del estacionamiento, identifica las problemáticas existentes y propone una solución basada en un sistema computacional. Incluye el diagnóstico de los procesos manuales actuales y la justificación de un sistema digital centralizado.}
	
\item {\textbf{Sección 2:} Introducción Proporciona un contexto general del proyecto,
incluyendo su propósito, alcance, definiciones clave, acrónimos, abreviaturas, referencias y esta visión general del contenido.}
\item {\textbf{Sección 3}: Descripción General Ofrece una visión panorámica del
sistema, incluyendo la perspectiva del producto, las funciones generales,
las características de los usuarios, restricciones, supuestos y dependencias,
así como los requisitos futuros identificados.}

\item {\textbf{Sección 4:} Requisitos Específicos Detalla las interfaces externas,
las funciones específicas que el sistema debe realizar, los requisitos de
rendimiento, restricciones de diseño, atributos del sistema y cualquier otro
requisito identificado.}
\item {\textbf{ Sección5}: Apéndices Contiene información adicional y complementaria
al documento, como diagramas, tablas y referencias técnicas, que pueden
ser útiles para la comprensión de los requisitos especificados.}
\end{itemize}

