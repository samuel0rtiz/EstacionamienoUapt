{\section{Descripción General del Documento}
	\subsection{Perspectiva del producto}
	El sistema \textbf{EstacionaUAPT} se desarrollará como una solución computacional destinada a optimizar la gestión del estacionamiento en la Unidad
	Académica Profesional Tianguistenco. Este sistema estará diseñado para reemplazar el proceso manual actual, centralizando los datos en una base de datos
	digital y proporcionando acceso en línea a usuarios y administradores. Se integra dentro del ecosistema tecnológico de la institución, con potencial para
	ampliarse en el futuro.
	\subsection{Funciones del producto}
	El sistema tendrá las siguientes funciones principales:
	\begin{itemize}
		\item Recepción y almacenamiento de datos de los usuarios (alumnos y personal
		administrativo).
		\item Gestión de solicitudes de acceso al estacionamiento.
		\item Generación de notificaciones automáticas para administradores y usuarios.
		\item Organización y búsqueda eficiente de registros en una base de datos centralizada.
		\item Emisión de comprobantes digitales para el acceso al estacionamiento.
		\item Registro y monitoreo de permisos de estacionamiento.
		\item Exportar datos de los usuarios a una tabla excel .

	\end{itemize}
	\subsection{Características de los usuarios}
	El sistema está dirigido a los siguientes tipos de usuarios:
	\begin{itemize}
		
\item {\textbf{Usuarios principales:} Estudiantes, docentes y personal administrativo
de la institución que requieren acceso al estacionamiento.}

\item {\textbf{Administrador}: Persona encargada de gestionar las solicitudes, validar los documentos y controlar los accesos al estacionamiento.}

	\end{itemize}
	Los usuarios principales tendrán acceso a un portal en línea para realizar sus
	solicitudes y consultar el estado de las mismas, mientras que el administrador
	contará con herramientas avanzadas para la gestión de datos y solicitudes.
	
	\subsection{Suposiciones y dependencias}
	El diseño y operación del sistema se basan en las siguientes suposiciones y dependencias:
\begin{itemize}
\item  Los usuarios disponen de acceso a dispositivos con conexión a internet.
\item  La institución proporcionará los recursos necesarios para la implementación
	del sistema, incluyendo servidores y bases de datos.
\item El sistema dependerá de servicios de autenticación y correo electrónico
	proporcionados por la institución.

\end{itemize}
\subsection{Requisitos futuros}
Se prevé que el sistema pueda evolucionar para incluir funcionalidades adi-
cionales, como:
\begin{itemize}

\item Integración con sistemas de pagos en línea para tarifas relacionadas con el
estacionamiento.
\item Expansión para la gestión de otros servicios, como el control de acceso a
edificios.
\item Incorporación de un módulo para el monitoreo en tiempo real de la ocu-
pación del estacionamiento.

\end{itemize}
Esta sección establece un marco general que facilita la definición de los req-
uisitos específicos en la siguiente sección.





}